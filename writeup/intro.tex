%%% Local Variables: 
%%% mode: latex
%%% TeX-master: "tellmemore"
%%% End: 

%{\bf Animal communication! Language!}

%--beginning froofroo
Warm summer nights in North America are often decorated with the mesmerizing light displays of fireflies. 
Flashes serve as the firefly's sexual signal during mate search and provide a proto-language for study. 

%--what are we looking at/previous bio studies

We consider the flash communication system of {\it Photinus} fireflies, a genus which contains 35 described species with simple on-off visual signals. 
The simplicity of these binary firefly communication signals lends themselves to study through simple computer modelling.

Previous studies have considered the dual evolutionary processes of sexual selection (mate choice) and natural selection (predation) with respect to the shaping of firefly flash signals. 
Fireflies use flashes to identify the species and sex of the signaler. 
Specifically, males, typically airborne, will signal with their species-specific patterns. 
Females, upon recognition of the correct signal, will respond to the male flashes with a species-specific response delay.
Based on the delay, males will either continue searching or approach physical contact, leading to mating. 

This simplistic overview of the courtship process highlights the necessity of distinguishable flash signals, as it is often the case that more than 10 flashing species of fireflies occupy a geographic area. 
While previous studies have considered selective pressures with respect to a specific species of firefly in shaping its own signal, it is unclear how these pressures interplay with the sympatric nature of the species. 
Indeed, Stanger-Hall and Lloyd (2015) conclude that reproductive character displacement was a main factor for signal divergence among sympatric {\it Photinus} species, but their findings fail to address the way in which that pressure shapes the signal itself. 

We thus seek to understand how various selection factors such as predation risk interact with the need for sympatric species to have mutually distinguishable signals and thus shape the evolution of firefly flash patterns. 
%how did firefly flash patterns evolve with respect to distinguishability?

