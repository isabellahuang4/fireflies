%%% Local Variables: 
%%% mode: latex
%%% TeX-master: "tellmemore"
%%% End: 

Upon visual inspection of the flash patterns, those patterns emerging from the partial sequence naming game seem to exhibit pseudo-periodic properties. 
Given that observed firefly flash signals are periodic in nature, that is, they can be fully described by the flash duration (length of a contiguous "on"), interpulse interval, and flash pattern interval, we proffer that periodic sequences are efficient solutions to the problem of distinguishability.
Moreover, the simulation selection pressures of interrupted signals as seen in the partial sequence naming game pose a better imitation of the selection pressures. 
This poses a question of how we might quantify efficiency with metrics other than simply counting the number of flashes in a pattern. 

We present the final patterns that emerge from the naming games after a few time steps of the game without individual patterns mutating. 
Tracking the evolution of patterns with respect to other patterns in all agents' memories might better inform how the space of patterns is searched and explored. 
This may lead to a better understanding of what types are patterns are ultimately fitter and more distinguishable. 

Moreover, if we track how patterns change throughout the naming game, it would be interesting to introduce, remove, and/or merge species during the game. 
Such a dynamic simulation would better highlight how the presence of a specific pattern affects distinguishability. 

%-- what happens if you introduce species in the middle of the naming game and/or remove species? how does this affect the trajectory of patterns
 
% periodic sequences are efficient solution to the problem???

